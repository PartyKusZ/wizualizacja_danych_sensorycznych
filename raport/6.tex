\section{Prezentacja wyników pracy - 10.05.2023r}
    Do dnia 10.05.2023r. dokonano następujących postępów:
        \begin{itemize}
            \item zaimplementowano odczyt danych z portu szeregowego, poprzez uruchomienie go w osobnym wątku. 
                Obiekt będący kontenerem na dane przekazywany jest przez wskaźnik do obiektu \textit{main\_window}, dzięki
                czemu dane są dostępne dla pozostałych komponentów aplikacji.
            \item zaprojektowano interfejs widoku ,,Temperatura''.
            \item zaprojektowano interfejs okienka służącego ustawianiu wartości alarmów,
            \item połączono sloty odpowiedzialne za aktualizacje właściwych im danych dotyczących elementów interfejsu, takich jak:
                \begin{itemize}
                    \item graficzna i tekstowa  prezentacja wypełnienia silosu (widok ,,Wszystkie parametry''),
                    \item graficzna i tekstowa prezentacja temperatury (widoki ,,Wszystkie parametry'' i ,,Temperatura''),
                    \item Prezentacja informacji o alarmach  (widoki ,,Wszystkie parametry'' i ,,Temperatura''),
                \end{itemize}
            \item przebudowano strukturę aplikacji, odciążono obiekt \textit{main\_window}, w którym znajdowały się 
                wszystkie sloty wykorzystywane przez aplikację. Utworzono klasy ,,backendowe'' w których usystematyzowano
                kod dotyczący  slotów.
        \end{itemize}