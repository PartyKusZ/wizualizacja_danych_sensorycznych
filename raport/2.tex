\section{Specyfikacja finalnego produktu} \label{ch: 2}
    Finalnym efektem projektu będzie aplikacja pozwalająca na monitorowanie parametrów
    wymienionych w pkt. \ref{ch: 1} i system czujników zastępujący prawdziwe silosy zbożowe.
    \subsection{Aplikacja} 
        \subsubsection{Wymagania funkcjonalne}
            Wymagania funkcjonalne są to wymagania które określają działanie systemu i zaspokajają potrzeby użytkownika.
            Poniżej znajduje się lista wymagań funkcjonalnych, które finalna wersja aplikacji powinna spełniać:
            \begin{itemize}
                \item Możliwość przeglądu monitorowanych na bieżąco parametrów stanu silosów:
                \begin{itemize}
                    \item każdego z parametrów osobno,
                    \item wszystkich parametrów razem.
                \end{itemize}
                \item prowadzenie rejestru pomiarów,
                \item wizualizacja pomiarów historycznych z określonego okresu czasu,
                \item ostrzeganie o zbliżaniu się do wartości niebezpiecznych,
                \item możliwość ustawienia wielkości wartości niebezpiecznych i parametrów związanych z alarmami,
                \item alarmowanie po przekroczeniu wartości niebezpiecznych.
            \end{itemize}
        \subsubsection{Wymagania niefunkcjonalne}
        Wymagania niefunkcjonalne określają przede wszystkim oczekiwania co do samej jakości działania 
        aplikacji oraz pożądanego zachowania tworzonego systemu.Poniżej znajduje się lista wymagań 
        niefunkcjonalnych, które finalna wersja aplikacji powinna spełniać:

        \begin{itemize}
            \item Użyte technologie: 
            \begin{itemize}
                \item C++17,
                \item Qt5,
            \end{itemize}
            \item możliwość zmiany rozmiaru ekranu i responsywność elementów GUI,
            \item wielojęzyczność,
            \item komunikacja z układem sensorów za pomocą portu szeregowego,
            \item przechowywanie danych historycznych w pliku CSV.
        \end{itemize}

    \subsection{System czujników}
        W celu realizowania odczytu z czujników zostanie skonstruowana prosta, niewielkich rozmiarów makieta silosów zbożowych, na której 
        zostaną osadzone odpowiednie czujniki:
        \begin{itemize}
            \item pomiary temperatury i wilgotności: DHT11 (4 sztuki, po na silos),
            \item pomiar wypełniania: HC SR04 (2 sztuki, po jednej na silos).
        \end{itemize}
    
    