\section{Charakterystyka tematu projektu} \label{ch: 1}
    Niniejszy projekt ma na celu stworzenie aplikacji służącej do wizualizacji silosów zbożowych. Aplikacja będzie wizualizować oraz 
    monitorować 3 kluczowe parametry dotyczące stanu silosów:
    \begin{itemize}
        \item wypełnienie,
        \item temperatura panująca wewnątrz,
        \item wilgotność panująca wewnątrz.
    \end{itemize} 
    Z perspektywy rolnika magazynującego zboże w silosach są to niezmiernie ważne informacje, dzięki nim będzie w stanie 
    w łatwy sposób szacować ilość zebranego plonu, monitorować wilgotność oraz temperaturę
    panującą w silosie, których zbyt wysokie wartości bardzo często są 
    wyznacznikiem tego, że w silosie rozpoczęły się procesy gnilne.
    \subsection{Główne cele aplikacji}
        Głównymi celami aplikacji będą:
        \begin{itemize}
            \item umożliwienie szybkiego i łatwego dostępu do informacji o stanie zboża w silosach,
            \item prezentowanie danych w przyjemniej i intuicyjnej formie graficznej,
            \item informowanie o zbliżaniu się do wartości krytycznych i przekroczeniu ich.
        \end{itemize}
    \subsection{Realizacja projektu}
            Aplikacja zostanie napisana w języku C++, wykorzystywać będzie bibliotekę Qt pozwalającą na tworzenie graficznego interfejsu użytkownika
            (GUI). Dane do wizualizacji udostępnianie będą przez zaprojektowany układ czujników, znajdujący się na makiecie silosów. 
    

   
